\documentclass[11pt]{article}
\usepackage[latin1]{inputenc}
\usepackage{amsmath, graphicx, layout, verbatim}
\usepackage{setspace}
\usepackage{fancyhdr}
\usepackage{float}
\usepackage{color}
\usepackage{natbib}
\usepackage{verbatim}
\usepackage{subfig}
\usepackage{amsthm}
\usepackage{amsfonts}
\usepackage{amssymb}
<<<<<<< HEAD
\usepackage[pagebackref,colorlinks=true,urlcolor=blue,citecolor=blue, linkcolor=blue]{hyperref}
=======
\usepackage[pagebackref,colorlinks=true,urlcolor=blue,citecolor=blue	,linkcolor=blue]{hyperref}
>>>>>>> 979001ef77a907f7b0beaeac363d4d8a5da11a18
\textheight25cm \footskip0.5cm \topmargin-1.5cm \headsep0.2cm
\oddsidemargin-0.5cm \evensidemargin-0.5cm
 \marginparwidth0cm \marginparsep = 0pt
\textwidth16cm

\onehalfspacing
\usepackage{cite}
\usepackage{url}
\urlstyle{rm}

\renewcommand{\vec}[1]{\mathbf{#1}}

\def\x{{\vec{x}}}


<<<<<<< HEAD
\def\G{{\mathbb G}}
 \newcommand\independent{\protect\mathpalette{\protect\independenT}{\perp}}
    \def\independenT#1#2{\mathrel{\setbox0\hbox{$#1#2$}%
    \copy0\kern-\wd0\mkern4mu\box0}}
=======

\def\G{{\mathbb G}}
 \newcommand\independent{\protect\mathpalette{\protect\independenT}{\perp}}
    \def\independenT#1#2{\mathrel{\setbox0\hbox{$#1#2$}%
    \copy0\kern-\wd0\mkern4mu\box0}} 
>>>>>>> 979001ef77a907f7b0beaeac363d4d8a5da11a18

 \def\X{***ATT***}
\newcommand{\red}[1]{\textbf{\color{red} ***ATT*** #1}}


\title{Response to Reviewer 2}
\date{}

\begin{document}

%the paper and
<<<<<<< HEAD
%hope to have answered the questions that were raised.
=======
%hope to have answered the questions that were raised. 
>>>>>>> 979001ef77a907f7b0beaeac363d4d8a5da11a18


\maketitle

\hspace{4mm} \textbf{Reviewer:} \textit{
<<<<<<< HEAD
    The authors explore the predictive performance of a few simple Multinomial-Dirichlet (MD)
    models relative to more sophisticated black box type procedures in predicting outcomes of
    football (soccer) matches in the second half of Brazil's division one. They find that under a
    variety of information metrics the simpler models perform similar to the more complicated ones.
    I think the paper is a good fit for the journal as it compares the predictive information gleaned
    from five models developed to predict football outcomes. However, there are a few issues I feel
    should be addressed. I organized my comments into two sections: general and specific.}
=======
	The authors explore the predictive performance of a few simple Multinomial-Dirichlet (MD)
	models relative to more sophisticated black box type procedures in predicting outcomes of
	football (soccer) matches in the second half of Brazil's division one. They find that under a
	variety of information metrics the simpler models perform similar to the more complicated ones.
	I think the paper is a good fit for the journal as it compares the predictive information gleaned
	from five models developed to predict football outcomes. However, there are a few issues I feel
	should be addressed. I organized my comments into two sections: general and specific.}
>>>>>>> 979001ef77a907f7b0beaeac363d4d8a5da11a18

\vspace{2mm}
\textbf{Response:} We thank the referee for taking the time reviewing the paper. Below we list our detailed response to the referee's questions.\\

\vspace{6mm}

\hspace{4mm} \textbf{Reviewer:} \textit{\textbf{General Comments.}
<<<<<<< HEAD
    What the authors have shown in my opinion is that the blackbox predictors they consider as
    competitors do not incorporate covariate information effectively and/or the covariates are not
    informative in learning about game outcomes. Even though it is not the focus of the paper,
    it seems to me that including some detail (at least on an intuitive level) of how information is
    used in these black box methods would be helpful in better understanding why they perform so
    poorly relative to the very unsophisticated MD models.}

\vspace{2mm} \textbf{Response: We included a new subsection
(Subsection 2.1) in Section 2 giving more details about the
benchmark models considered, which are quite similar to many other
approaches in the literature. In fact, we believe that soccer
prediction is really a hard task. This is mainly because several
factors may have a great influence on the final match outcome such
as referee's error, receiving a red card, injury of important
players, weather, among many others. All those factors are difficult
(maybe impossible) to take into account in any statistical model.
Thus, the use of elaborated models leads to small gain in prediction
accuracy when compared to more simpler models, because of the great
unpredictability of soccer matches.}
=======
	What the authors have shown in my opinion is that the blackbox predictors they consider as
	competitors do not incorporate covariate information effectively and/or the covariates are not
	informative in learning about game outcomes. Even though it is not the focus of the paper,
	it seems to me that including some detail (at least on an intuitive level) of how information is
	used in these black box methods would be helpful in better understanding why they perform so
	poorly relative to the very unsophisticated MD models.}

\vspace{2mm}
\textbf{Response: we added subsection in Section 2 explaining some details of the benchmark models.***Falta o Ernesto***.} 
>>>>>>> 979001ef77a907f7b0beaeac363d4d8a5da11a18


%We believe this may the case because
%of the fact that estimating $\psi_{j}(\vec{x})$'s accurately is relatively hard (see, e.g., Lee and Wasserman 2010).
<<<<<<< HEAD
%Another related issue is that, as noticed by the referee, although $\psi_{j}(\vec{x})$
%are orthonormal,
%$\widehat{\psi}_{j}(\vec{x})$ are not.
=======
%Another related issue is that, as noticed by the referee, although $\psi_{j}(\vec{x})$ 
%are orthonormal,
%$\widehat{\psi}_{j}(\vec{x})$ are not. 
>>>>>>> 979001ef77a907f7b0beaeac363d4d8a5da11a18

\vspace{6mm}

\hspace{4mm} \textbf{Reviewer:} \textit{On a related note (and something I discuss a bit more later), it wasn?t entirely clear to me how
<<<<<<< HEAD
    information was used to make game predictions for the black box methods vs the MD models. I
    understand that predictions from the MD model were done sequentially one at a time employing
    the Bayesian method of updating information as more game outcomes where considered. Is this
    what was done with the black box methods? Or where all the second half games predicted only
    using the first games. I think trying to identify why the MD models are performing on par with
    more sophisticated alternatives would be interesting.
}

\vspace{2mm}
\textbf{Response: we added two paragraphs in subsection 2.2 explaining how the models we are proposing were updated. We used all past results of the teams playing against each other up to the matchday that is being predicted. In this sense, all predictions are ``one-step ahead'' predictions. Regarding the benchmark models, subsection 2.1 comments how they use the information to predict the results. Basically, the predictions of the benchmark models are also one-step ahead predictions, even though they could use the information of the first half of the championship to predict all the second half.}
=======
	information was used to make game predictions for the black box methods vs the MD models. I
	understand that predictions from the MD model were done sequentially one at a time employing
	the Bayesian method of updating information as more game outcomes where considered. Is this
	what was done with the black box methods? Or where all the second half games predicted only
	using the first games. I think trying to identify why the MD models are performing on par with
	more sophisticated alternatives would be interesting.
}

\vspace{2mm}
\textbf{Response: we added two paragraphs in subsection 2.2 explaining how the models we are proposing were updated. We used all past results of the teams playing against each other up to the matchday that is being predicted. In this sense, all predictions are ``one-step ahead'' predictions. Regarding the benchmark models, subsection 2.1 comments how they use the information to predict the results. Basically, the predictions of the benchmark models are also one-step ahead predictions, even though they could use the information of the first half of the championship to predict all the second half.} 
>>>>>>> 979001ef77a907f7b0beaeac363d4d8a5da11a18
% \red{I'm not sure!! Does he mean $1/(\lambda_j\Delta_J^2)$ has to be bounded by the same constant for all $j$'s? Recall that $%\lambda_1>\ldots>\lambda_J$ and $\Delta_1 > \ldots > \Delta_J$}

\vspace{6mm}

<<<<<<< HEAD
\hspace{4mm} \textbf{Reviewer:} \textit{\textbf{Section 1: Introduction.}
    \begin{itemize}
        \item Line 17 page 2 should ``provisions" be ``predictions"?
        \item It may be beneficial to the reader to briefly describe the second half of the Brazilian
        championship. (As an aside, on page 4 instead of using `` second half", ``second round"
        is used. Are you referring to the same set of games?) How many games per team in the
        second half?
    \end{itemize}}

\vspace{2mm}
\textbf{Response: we changed ``previsions'' by ``predictions''. There was a mistake on p. 4 because we should refer to the ``second round''. We fixed this and added a noted explaining that each team plays 19 matches each round, since the championship has 20 teams. More details on the championship rule are given in Section 4 (Results).}

\vspace{6mm}

\hspace{4mm} \textbf{Reviewer:} \textit{\textbf{Section 2: The models: theoretical background.}
    \begin{itemize}
        \item You point out that the assumptions associated for the MD models and black-box models
        are different. Does this make comparing them untenable? If not, why?
    \end{itemize}}

    \vspace{2mm}
    \textbf{Response: We believe that in the such cases where the model assumptions are quite different, the only possible comparison are in terms
    of predictive accuracy. In the present study, this comparison is possible since all the models report the probabilities of win, draw or loss for the home team.
    Furthermore, from the perspective of soccer fans the most important value of a model is giving ``good'' predictions.}

\vspace{6mm}

\hspace{4mm} \textbf{Reviewer:} \textit{\textbf{Section 2.1: Multinomial-Dirichlet.}
    \begin{itemize}
        \item In line 8, ``belongs" is misspelled.
        \item The first paragraph of page 4 is somewhat misleading. If I am understanding things cor-
        rectly, predicting the outcome of game $X_{n+m+1}$ is done using games $X_{n+m} ,\ldots , X_m , \ldots , X_1$
        (where games $X_m , \ldots , X_1$ denote the first half games) and are carried out by repeatedly
        using Bayes information updating mechanism (posterior from old study becomes prior for
        new). Therefore, in reality your prior is not $D(1, 1, 1)$ but rather the posterior distribu-
        tion from outcomes of the first half games coupled a $D(1, 1, 1)$ prior. Can you clarify or
        comment on this? Is this how the blackbox methods are being used as well?
    \end{itemize}}

    \vspace{2mm}
    \textbf{Response: we fixed the word ``belongs'', now on subsection 2.2. Regarding the updating of information, the reviewer understood correctly. We rewrote the first paragraph of p.4 to explain the operation in detail. Regarding the blackbox models, subsection 2.1 clarifies how they use available information.}

\vspace{6mm}

\hspace{4mm} \textbf{Reviewer:} \textit{\textbf{Section 2.4: Model 3: Multinomial-Dirichlet 3.}
    \begin{itemize}
        \item The section title should be ``Model Three:" (spell out the number 3) in order to be
        consistent with outher section headings.
    \end{itemize}}

    \vspace{2mm}
    \textbf{Response: now we spell out 3, as suggested.}


    \vspace{6mm}

    \hspace{4mm} \textbf{Reviewer:} \textit{\textbf{Section 3: Scoring rules and calibration.}
        \begin{itemize}
            \item The word ``prevision" appears a few times in the first four paragraphs of this section. I
            don't understand it's use. I initially thought ``prediction" was implied, but am not sure.
            \item Related to the previous comment, I understand that $P_X$ is a probability not a prediction of
            say $X_{n+1}$. Predictions of $X_{n+1}$ should live in $X = \{1, 2, 3\}$. Consider revising terminology.
            In reality the first five paragraphs of Section 3 are hard to read. I'd suggest revising
            exposition.
        \end{itemize}}

        \vspace{2mm}
        \textbf{Response: the word ``prevision'' was replaced by ``prediction''. The first paragraphs were all rewritten to better explain the concepts we introduce. Terminology was also reviewed, as suggested.}


\vspace{6mm}

We are grateful to the referee for the insightful comments that helped to improve the quality of our paper.
We expect to have addressed all his/her concerns. Please feel
free to contact us in case something is still not clear, or if new concerns arise.

\end{document}
=======
\hspace{4mm} \textbf{Reviewer:} \textit{\textbf{Section 1: Introduction.} 
	\begin{itemize}
		\item Line 17 page 2 should ``provisions" be ``predictions"?
		\item It may be beneficial to the reader to briefly describe the second half of the Brazilian
		championship. (As an aside, on page 4 instead of using `` second half", ``second round"
		is used. Are you referring to the same set of games?) How many games per team in the
		second half?
	\end{itemize}}

\vspace{2mm}
\textbf{Response: we changed ``previsions'' by ``predictions''. There was a mistake on p. 4 because we should refer to the ``second round''. We fixed this and added a noted explaining that each team plays 19 matches each round, since the championship has 20 teams. More details on the championship rule are given in Section 4 (Results).} 

\vspace{6mm}

\hspace{4mm} \textbf{Reviewer:} \textit{\textbf{Section 2: The models: theoretical background.} 
	\begin{itemize}
		\item You point out that the assumptions associated for the MD models and black-box models
		are different. Does this make comparing them untenable? If not, why?
	\end{itemize}}
	
	\vspace{2mm}
	\textbf{Response:***Ernesto, pode ajudar?***} 
	
\vspace{6mm}

\hspace{4mm} \textbf{Reviewer:} \textit{\textbf{Section 2.1: Multinomial-Dirichlet.} 
	\begin{itemize}
		\item In line 8, ``belongs" is misspelled.
		\item The first paragraph of page 4 is somewhat misleading. If I am understanding things cor-
		rectly, predicting the outcome of game $X_{n+m+1}$ is done using games $X_{n+m} ,\ldots , X_m , \ldots , X_1$
		(where games $X_m , \ldots , X_1$ denote the first half games) and are carried out by repeatedly
		using Bayes information updating mechanism (posterior from old study becomes prior for
		new). Therefore, in reality your prior is not $D(1, 1, 1)$ but rather the posterior distribu-
		tion from outcomes of the first half games coupled a $D(1, 1, 1)$ prior. Can you clarify or
		comment on this? Is this how the blackbox methods are being used as well?
	\end{itemize}}
	
	\vspace{2mm}
	\textbf{Response: we fixed the word ``belongs'', now on subsection 2.2. Regarding the updating of information, the reviewer understood correctly. We rewrote the first paragraph of p.4 to explain the operation in detail. Regarding the blackbox models, subsection 2.1 clarifies how they use available information.} 
	
\vspace{6mm}

\hspace{4mm} \textbf{Reviewer:} \textit{\textbf{Section 2.4: Model 3: Multinomial-Dirichlet 3.} 
	\begin{itemize}
		\item The section title should be ``Model Three:" (spell out the number 3) in order to be
		consistent with outher section headings.
	\end{itemize}}
	
	\vspace{2mm}
	\textbf{Response: now we spell out 3, as suggested.} 
	
	
	\vspace{6mm}
	
	\hspace{4mm} \textbf{Reviewer:} \textit{\textbf{Section 3: Scoring rules and calibration.} 
		\begin{itemize}
			\item The word ``prevision" appears a few times in the first four paragraphs of this section. I
			don't understand it's use. I initially thought ``prediction" was implied, but am not sure.
			\item Related to the previous comment, I understand that $P_X$ is a probability not a prediction of
			say $X_{n+1}$. Predictions of $X_{n+1}$ should live in $X = \{1, 2, 3\}$. Consider revising terminology.
			In reality the first five paragraphs of Section 3 are hard to read. I'd suggest revising
			exposition.
		\end{itemize}}
		
		\vspace{2mm}
		\textbf{Response: the word ``prevision'' was replaced by ``prediction''. The first paragraphs were all rewritten to better explain the concepts we introduce. Terminology was also reviewed, as suggested.} 
		
		
\vspace{6mm}

We are grateful to the referee for the insightful comments that helped to improve the quality of our paper. 
We expect to have addressed all his/her concerns. Please feel
free to contact us in case something is still not clear, or if new concerns arise.

\end{document}
>>>>>>> 979001ef77a907f7b0beaeac363d4d8a5da11a18
